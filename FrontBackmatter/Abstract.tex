%*******************************************************
% Abstract
%*******************************************************
%\renewcommand{\abstractname}{Abstract}
\pdfbookmark[1]{Abstract}{Abstract}
\begingroup
\let\clearpage\relax
\let\cleardoublepage\relax
\let\cleardoublepage\relax

\chapter*{Abstract}

Visual localisation and object recognition might seem to be diverging topics in recent times, with localisation being more focused on geometry inference and mapping and object recognition research pushing the boundaries towards the big data domain and making use of deep learning methods. However, we believe there are areas of overlap if a particular type of methods are utilised. Appearance-based methods fall in this category and it is the main purpose of this thesis to explore them in the novel contexts of wearable and hand-held object recognition and visual localisation. Additionally, a particular point of emphasis of this thesis is on whether biologically-inspired models can be applied to these appearance-based algorithms and whether this biological mimicry provides a positive impact on performance.

Therefore, around this topic we have elaborated individual research hypotheses for each of the topics developed in each chapter. The quest for answers to these hypotheses has produced several contributions. In first place, we have collected two large datasets and studied the constraints of the data problem in the cases of object recognition and indoor visual localisation. In second place, using these datasets and those provided by the community, we constructed evaluation pipelines for object recognition and visual localisation; and benchmarked custom-created image desription methods with baseline and state-of-the-art ones for performance in this context. In third, place, we have developed a model of the biologicall place cells, the artificial place cells, and tested their performance under the challenging conditions of indoor localisation, resulting in a successful approach even against the state of the art SLAM, a definitely mature technology. Finally, we have prototyped an assistive localisation system using wearable or hand-held visual input and tactile feedback to track the localisation of the user over haptic maps.


\vfill

\pdfbookmark[1]{Resumen}{Resumen}
\chapter*{Resumen}
Abstract in Spanish\dots


\endgroup			

\vfill