%*******************************************************
% Table of Contents
%*******************************************************
%\phantomsection
\refstepcounter{dummy}
\pdfbookmark[1]{\contentsname}{tableofcontents}
\setcounter{tocdepth}{2} % <-- 2 includes up to subsections in the ToC
\setcounter{secnumdepth}{3} % <-- 3 numbers up to subsubsections
\manualmark
\markboth{\spacedlowsmallcaps{\contentsname}}{\spacedlowsmallcaps{\contentsname}}
\tableofcontents 
\automark[section]{chapter}
\renewcommand{\chaptermark}[1]{\markboth{\spacedlowsmallcaps{#1}}{\spacedlowsmallcaps{#1}}}
\renewcommand{\sectionmark}[1]{\markright{\thesection\enspace\spacedlowsmallcaps{#1}}}
%*******************************************************
% List of Figures and of the Tables
%*******************************************************
\clearpage

\begingroup 
    \let\clearpage\relax
    \let\cleardoublepage\relax
    \let\cleardoublepage\relax
    %*******************************************************
    % List of Figures
    %*******************************************************    
    %\phantomsection 
    \refstepcounter{dummy}
    %\addcontentsline{toc}{chapter}{\listfigurename}
    \pdfbookmark[1]{\listfigurename}{lof}
    \listoffigures

    \vspace*{8ex}

    %*******************************************************
    % List of Tables
    %*******************************************************
    %\phantomsection 
    \refstepcounter{dummy}
    %\addcontentsline{toc}{chapter}{\listtablename}
    \pdfbookmark[1]{\listtablename}{lot}
    \listoftables
        
    \vspace*{8ex}
%   \newpage
    
    %*******************************************************
    % List of Listings
    %*******************************************************      
	  %\phantomsection 
    %\refstepcounter{dummy}
    %\addcontentsline{toc}{chapter}{\lstlistlistingname}
    %\pdfbookmark[1]{\lstlistlistingname}{lol}
    %\lstlistoflistings 

    %\vspace*{8ex}
	\newpage       
    %*******************************************************
    % Acronyms
    %*******************************************************

    \phantomsection 
    \refstepcounter{dummy}
    \pdfbookmark[1]{Key terms}{terms}
    \markboth{\spacedlowsmallcaps{Key terms}}{\spacedlowsmallcaps{Key terms}}
    \chapter*{Key terms}
    \begin{acronym}[UML]
        \acro{Features}{A feature is a property of an image that is used to solve a computational task. An example of a feature can be an interesting point, an edge or a shape present in the image.}
        \acro{Keypoints}{Also called interesting points, keypoints are a specific type of features that capture special local properties around a coordinate point within the image.}
        \acro{Descriptors}{They are descriptions of a visual feature present in the image such as colour, texture or shape. They are related to keypoints as these can be used as locations for the computation of descriptors that measure local properties of the image.}
        \acro{Visual Path}{A collection of image frames that are induced by the relative motion of a person in a scene.}
		\acro{Journey}{In our context, a visual path that has a start 'A' and end 'B' points.}
        \acro{Corridor}{In the present thesis, a corridor represents a the recording of a segment of a journey that in great proportion traverses a corridor within a building.}
        \acro{Pass}{Each of the recording instances of the same corridor in the RSM dataset.}
        \acro{Biological Place Cells}{BPCs are a specific type of neuron found in mammals that exhibit an increased firing rate when the subject navigates a previously visited place.}
        \acro{Artificial Place Cells}{The computational models of the BPCs presented in this thesis.}
    \end{acronym}                     
\endgroup

\cleardoublepage