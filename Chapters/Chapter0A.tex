%********************************************************************
% Appendix
%*******************************************************
% If problems with the headers: get headings in appendix etc. right
%\markboth{\spacedlowsmallcaps{Appendix}}{\spacedlowsmallcaps{Appendix}}
\chapter{Appendices}


\newpage
\section{Algorithm for generating cumulative error distributions}
\label{appendixCDF}
\label{appendixCDF}

{\fontsize{10}{10}\selectfont
\begin{algorithm}[H]
\DontPrintSemicolon
\SetKwInOut{Input}{Inputs}
\SetKwInOut{Output}{Outputs}
\SetKwFunction{getClosestNeighbor}{getClosestNeighbor}
\SetKwFunction{computeError}{computeError}
\SetKwFunction{randomSampling}{randomSampling}
\SetKwFunction{computeCDF}{computeCDF}

\Input{\;
	\Indp \Indp 	Database of kernels, 	 $ \mathcal{K}_{c,p,l}$\;
	$c = 1, 2, ..., N_c $, \tcp*[l]{corridor index} \;
	$ p = 1, 2, ..., N_p $ \tcp*[l]{pass index}\;
	Number of permutations, $P$ \;
	Number of random queries, $Q$
	
}

\Output{\;
\Indp \Indp Error Distribution,	 $\mathbf{X}$} 

\BlankLine

\tcp*[l]{Compute localisation error for all possible queries}		 

\For{$c \leftarrow 1$ \KwTo $N_c$}{
	\For{$p \leftarrow 1$ \KwTo $N_p$} {
	 \tcp*[l]{For each query frame in a pass ...}
		 \ForEach{$q: q \in \mathcal{P}_p$}{ 
		 \tcp*[l]{Take the corresponding kernel computed by leave-one-out strategy and get closest neighbour}
		 %\tcp*[l]{... and get closest neighbour}

		 $\rho \leftarrow \getClosestNeighbor(K)$\;
		 \BlankLine
		 \tcp*[l]{Given the ground truth for that query, compute the error}		 

		 $\mathbf{E}_{c,p,q} \leftarrow \computeError(\rho)$  
}
}
}

$k \leftarrow 1$ 
\BlankLine
\For{$i \leftarrow 1$ \KwTo $P$} {
		 \For{$j: j \leftarrow 1$ \KwTo $Q$} {
		 $e_k \leftarrow randomSampling(E)$\;
		 $k \leftarrow k + 1$
		 
}
} 
\tcp*[l]{Compute Cumulative Distribution Functions}
$\mathbf{X} \leftarrow \computeCDF(e_k)$
\caption{Calculation of the error distribution}
\label{algo:error_distr}
\end{algorithm}
}

\newpage

\section{Tactile feedback experiment protocol}
\subsection{Context}
The ``Visual localization with tactile feedback'' project aims to evaluate the quality of the tactile feedback given by the Senseg tablet in an indoor localization for the visually impaired context. When navigating a physical path, the user receives a tactile cue that encodes an estimate of their position along that specific path, relative to start and end point. Given several location estimate feedback cues through the Senseg tactile interface, the goal of this experiment is to evaluate how accurate this tactile feedback is based on the user perception of the position they are.

\subsection{Experiment protocol}

\begin{enumerate}

\item The user will be given some familiarization tasks with the Senseg demo that shipped with the tablet as Android applications. These will be:

\begin{itemize}
\item Familiarize with the different textures with the app ``Haptic Guidelines''.
\end{itemize}

\item The user will be given the following instructions:

\begin{framed}
You have agreed to take part in the ``Visual localization with tactile feedback" project experiment on tactile feedback quality. The experiment consists of the following tasks:

\begin{enumerate}
\item You will be given the Senseg tablet you used previously to get familiar with its tactile interface.

\item If you visually inspect the path, you will notice 2 red rectangles that denote starting and end point. These are texture highlighted. As you feel the screen with your finger and move it over the path you will notice four haptic ``landmarks'' or ``events'' that can be differentiated: 

\begin{enumerate}
\item the beginning of the path,
\item an area with no haptic feedback, this is the area that would represent the area that users have already traversed,
\item an area with haptic feedback, that represents the remaining segment of the path,
\item the end of the path, with highlighted haptic texture as event (i).
\end{enumerate}
\item You will receive one tactile cue every 15 seconds, making up to 20 cues. 

\item Upon the reception of the cue, it will be your task to announce an estimate of your location as a percentage of the total distance. You will only provide estimates that are 10\% apart:
\begin{itemize}
\item 0\%: starting point of the journey,
\item 10\%
\item 20\%,
\item ...
\item 80\%,
\item 90\%
\item 100\%: end point of the journey.
 
\end{itemize}

\item As agreed, you will blindfold yourself for this experiment. Please, proceed to wear the blindfold now, the experiment will start shortly.
\end{enumerate}
\end{framed}

\item The experiment will start:
\begin{enumerate}
\item The user will receive 20 tactile cues corresponding to 20 randomized location estimates provided by the localization server.
\item The users' announced estimates will be annotated next to their corresponding index in the following table\footnote{Not present in the volunteer copy.}


\begin{table}[h]
\centering
    \begin{tabular}{|c|c|c|}
    \hline Trial index & True location & Estimated location  \\ \hline
    1           & ~       0.81        	  &    \\ \hline
    2           & ~       0.91             &    \\ \hline
    3           & ~       0.13             &    \\ \hline
    4           & ~       0.91             &    \\ \hline
    5           & ~       0.63             &    \\ \hline
    6           & ~       0.10             &    \\ \hline
    7           & ~       0.28             &    \\ \hline
    8           & ~       0.55             &    \\ \hline
    9           & ~       0.96             &    \\ \hline
    10          & ~       0.16             &    \\ \hline
    11          & ~       0.49             &    \\ \hline
    12          & ~       0.80             &    \\ \hline
    13          & ~       0.14             &    \\ \hline
    14          & ~       0.42             &    \\ \hline
    15          & ~       0.79             &    \\ \hline
    16          & ~       0.66             &    \\ \hline
    17          & ~       0.04             &    \\ \hline
    18          & ~       0.93             &    \\ \hline
    19          & ~       0.40             &    \\ \hline
    20          & ~       0.27             &    \\ \hline
    \end{tabular}
\end{table}

\end{enumerate} 

\end{enumerate}

\newpage

\subsection{Informed consent form}

\subsubsection{Experiment purpose \& procedure}

The purpose of this experiment is to evaluate the quality of the tactile feedback given by the Senseg tablet in an indoor localization for the visually impaired context.

The experiment consists of 2 parts as detailed in the previous section

After the experiment, you will be asked to complete a feedback form.

Please note that none of the tasks is a test of your personal intelligence or ability. The objective is to test the usability of our research systems

\subsubsection{Confidentiality}

The following data will be recorded: Estimates of the tactile-encoded position along a path based on Senseg haptic feedback.

All data will be coded so that your anonymity will be protected in any research papers and presentations that result from this work.

\subsubsection{Finding out about result}

If interested, you can find out the result of the study by contacting the researcher Jose Rivera-Rubio, after 1 April 2015. 

His email address is jose.rivera@imperial.ac.uk.

\subsubsection{Record of consent}
Your signature below indicates that you have understood the information about the ``Tactile feedback with Senseg'' experiment and consent to your participation. The participation is voluntary and you may refuse to answer certain questions on the questionnaire and withdraw from the study at any time with no penalty. This does not waive your legal rights. You should have received a copy of the consent form for your own record. If you have further questions related to this research, please contact the researcher.


\def\info#1\par{#1\hrulefill\hss\par}
\everypar={\setbox0=\lastbox}
\baselineskip=15pt
\hsize=.5\hsize
\info Your Name

\info Your Signature

\vspace{1cm}

Researcher: Jose Rivera-Rubio


Date: 10/03/2015


\newpage
\section{LSD-SLAM parameters}
\label{sec:slampars}

\begin{table}[h]
\centering
    \begin{tabular}{L{2cm} L{9cm} c c}
    \hline
    Parameter                    & Definition                                                                                                                                                                                                                                                                            & Default & Set to \\ \hline
    minUserGrad                  & Minimal absolute image gradient for a pixel to be used at all. Increase if your camera has large image noise, decrease if you have low image-noise and want to also exploit small gradients.                                                                                          & 1.96    & 5      \\ \hline
    cameraPixelNoise             & Image intensity noise used for e.g. tracking weight calculation. Should be set larger than the actual sensor-noise, to also account for noise originating from discretization / linear interpolation.                                                                                 & 16      & 2.4    \\ \hline
    KFUsage-Weight                & Determines how often keyframes are taken, depending on the overlap to the current keyframe. Larger: more keyframes.                                                                                                                                                                 & 4       & 10     \\ \hline
    KFDistWeight                 & Determines how often keyframes are taken, depending on the distance to the current Keyframe. Larger: more keyframes.                                                                                                                                                                & 3       & 10     \\ \hline

    \end{tabular}
\end{table}

